% Thorn documentation template
\documentclass{article}
\begin{document}

\title{IOASCII}
\author{Thomas Radke, Gabrielle Allen}
\date{December 2001}
\maketitle

\abstract{Thorn IOASCII provides I/O methods for 1D, 2D, and 3D output of
grid arrays into files in ASCII format. The precise format is designed for
visualisation using the clients {\tt xgraph} or {\tt gnuplot}.}

\section{Purpose}
Thorn IOASCII registers three I/O methods named {\tt IOASCII\_1D}, {\tt IOASCII\_2D}, and
{\tt IOASCII\_3D} with the I/O interface in the flesh.

\begin{itemize}

  \item {\tt IOASCII\_1D} 

  creates one-dimensional output of 1D, 2D and
  3D grid functions and arrays by slicing through the edge (in the
  octant case) or center (in all origin centered cases) of the grid in
  the coordinate directions.  In addition, output is provided
  along a diagonal of the grid, in this case the diagonal always starts
  at the first grid point (that is, in Fortran notation {\tt var(1,1,1)})
  and the line taken uses grid points increasing by 1 in each direction.
  [NOTE: The diagonal output is not yet available for staggered variables].

  Output for each direction can be
  selected individually via parameters.

  Data is written in ASCII format and goes into files named

  {\tt <GF\_name>.[xyzd]}

  with the default behaviour in each grid dimension being to create
  the following files:

  \begin{center}
  \begin{tabular}{cc}
    Array dimension & Default output \\
    3D & {\tt xl yl zl dl} \\
    2D & {\tt xl yl dl} \\
    1D & {\tt xl} 
  \end{tabular}
  \end{center}

  These files can be processed directly by either xgraph or gnuplot
  (you can select the style of output via parameter settings).

  \item {\tt IOASCII\_2D} 

  outputs two-dimensional slices of grid functions and arrays planes. 
  Again, slicing is done through the edge (in the
  octant case) or center (in all origin centered cases).\\ Data is
  written in ASCII format and goes into files named 
  
  {\tt <varname>\_2d\_[\{xy\}\{xz\}\{yz\}].gnuplot}

  The default behaviour in each grid dimension is to create
  the following files:

  \begin{center}
  \begin{tabular}{cc}
    Array dimension & Default output \\
    3D & {\tt xy yz xz} \\
    2D & {\tt xy} 
  \end{tabular}
  \end{center}

  These files can be
  visualized by gnuplot using its {\it splot} command.  

  \item {\tt IOASCII\_3D} 

  outputs three-dimensional grid functions and arrays as
  a whole.

  Data is written in ASCII format and goes into files named
  
  {\tt <varname>\_3D.asc}

  These files can be visualized by gnuplot
  using its {\it splot} command.

\end{itemize}
%
You obtain output by an I/O method by either
%
\begin{itemize}
  \item setting the appropriate I/O parameters
  \item calling one the routines of the I/O function interface provided by the flesh
\end{itemize}
%
For a description of basic I/O parameters and the I/O function interface to
invoke I/O methods by application thorns please see the documentation of thorn
IOUtil and the flesh.
%
\section{Comments}

IMPORTANT: Must select data to output AFTER spatial coordinates are set up




%
Since IOASCII uses parameters from IOUtil
it also needs this I/O utility thorn be compiled into Cactus and activated.

\section{Examples}

In this section we include example output for different parameter combinations. 
Note that all these examples were generated for just a couple of timesteps for an extremely small 3D grid.

\subsection{One dimensional xgraph}

These options produce data suitable for using with the xgraph visualization client 
in the format 
{\tt
\begin{verbatim}
x f(t=fixed,x,y=fixed,z=fixed)
\end{verbatim}
}


{\tt
\begin{verbatim}
IOASCII::out1D_every = 1
IOASCII::out1D_vars = "wavetoy::phi"
IOASCII::out1D_style = "xgraph"
\end{verbatim}
}

\noindent
{\bf Output File: phi.xl}

{\tt
\begin{verbatim}
"Parameter file wavetoy.par
"Created Sun 19 Aug 2001 16:31:43
"x-label x
"y-label WAVETOY::phi (y = 0.0000000000000, z = 0.0000000000000), (yi = 1, zi = 1) 


"Time = 0.0000000000000
-0.5000000000000		0.0000000000139
0.0000000000000		1.0000000000000
0.5000000000000		0.0000000000139


"Time = 0.2500000000000
-0.5000000000000		0.0000000000000
0.0000000000000		0.4980695458846
0.5000000000000		0.0000000000000


"Time = 0.5000000000000
-0.5000000000000		0.0019304541362
0.0000000000000		-0.7509652270577
0.5000000000000		0.0019304541362
\end{verbatim}
}

\subsection{One dimensional gnuplot}

These options produce data suitable for using with the gnuplot visualization client in the format
{\tt
\begin{verbatim}
x f(t,x,y=fixed,z=fixed)
\end{verbatim}
}


{\tt
\begin{verbatim}
IOASCII::out1D_every = 1
IOASCII::out1D_vars = "wavetoy::phi"
IOASCII::out1D_style = "gnuplot f(x)"
\end{verbatim}
}

\noindent
{\bf Output File: phi.xl}

{\tt
\begin{verbatim}
#Parameter file wavetoy.par
#Created Sun 19 Aug 2001 16:33:07
#x-label x
#y-label WAVETOY::phi (y = 0.0000000000000, z = 0.0000000000000), (yi = 1, zi = 1) 

#Time = 0.0000000000000
-0.5000000000000		0.0000000000139
0.0000000000000		1.0000000000000
0.5000000000000		0.0000000000139

#Time = 0.2500000000000
-0.5000000000000		0.0000000000000
0.0000000000000		0.4980695458846
0.5000000000000		0.0000000000000

#Time = 0.5000000000000
-0.5000000000000		0.0019304541362
0.0000000000000		-0.7509652270577
0.5000000000000		0.0019304541362
\end{verbatim}
}


\subsection{One dimensional gnuplot (including time)}

These options produce data suitable for using with the gnuplot visualization client in the format
{\tt
\begin{verbatim}
t x f(t,x,y=fixed,z=fixed)
\end{verbatim}
}

{\tt
\begin{verbatim}
IOASCII::out1D_every = 1
IOASCII::out1D_vars = "wavetoy::phi"
IOASCII::out1D_style = "gnuplot f(t,x)"
\end{verbatim}
}

\noindent
{\bf Output file: phi.xl}
{\tt
\begin{verbatim}
#Parameter file wavetoy.par
#Created Sun 19 Aug 2001 16:34:48
#x-label x
#y-label WAVETOY::phi (y = 0.0000000000000, z = 0.0000000000000), (yi = 1, zi = 1) 

#Time = 0.0000000000000
0.0000000000000		-0.5000000000000		0.0000000000139
0.0000000000000		0.0000000000000		1.0000000000000
0.0000000000000		0.5000000000000		0.0000000000139

#Time = 0.2500000000000
0.2500000000000		-0.5000000000000		0.0000000000000
0.2500000000000		0.0000000000000		0.4980695458846
0.2500000000000		0.5000000000000		0.0000000000000

#Time = 0.5000000000000
0.5000000000000		-0.5000000000000		0.0019304541362
0.5000000000000		0.0000000000000		-0.7509652270577
0.5000000000000		0.5000000000000		0.0019304541362
\end{verbatim}
}

\subsection{Two dimensional gnuplot}

These options produce data suitable for using with the gnuplot visualization client in the format
{\tt
\begin{verbatim}
x y f(t,x,y,z=fixed)
\end{verbatim}
}


{\tt
\begin{verbatim}
IOASCII::out2D_every = 1
IOASCII::out2D_vars = "wavetoy::phi"
IOASCII::out2D_style = "gnuplot f(x,y)"
\end{verbatim}
}

\noindent
{\bf Output file: phi\_2d\_xy.gnuplot}
{\tt
\begin{verbatim}
#Parameter file wavetoy.par
#Created Sun 19 Aug 2001 16:31:43
#x-label x
#y-label y
#z-label WAVETOY::phi (z = 0.0000000000000), (zi = 1)


#Time = 0.0000000000000
-0.5000000000000		-0.5000000000000		0.0000000000000
0.0000000000000		-0.5000000000000		0.0000000000139
0.5000000000000		-0.5000000000000		0.0000000000000

-0.5000000000000		0.0000000000000		0.0000000000139
0.0000000000000		0.0000000000000		1.0000000000000
0.5000000000000		0.0000000000000		0.0000000000139

-0.5000000000000		0.5000000000000		0.0000000000000
0.0000000000000		0.5000000000000		0.0000000000139
0.5000000000000		0.5000000000000		0.0000000000000



#Time = 0.2500000000000
-0.5000000000000		-0.5000000000000		0.0000000000000
0.0000000000000		-0.5000000000000		0.0000000000000
0.5000000000000		-0.5000000000000		0.0000000000000

-0.5000000000000		0.0000000000000		0.0000000000000
0.0000000000000		0.0000000000000		0.4980695458846
0.5000000000000		0.0000000000000		0.0000000000000

-0.5000000000000		0.5000000000000		0.0000000000000
0.0000000000000		0.5000000000000		0.0000000000000
0.5000000000000		0.5000000000000		0.0000000000000



#Time = 0.5000000000000
-0.5000000000000		-0.5000000000000		0.0000000008425
0.0000000000000		-0.5000000000000		0.0019304541362
0.5000000000000		-0.5000000000000		0.0000000008425

-0.5000000000000		0.0000000000000		0.0019304541362
0.0000000000000		0.0000000000000		-0.7509652270577
0.5000000000000		0.0000000000000		0.0019304541362

-0.5000000000000		0.5000000000000		0.0000000008425
0.0000000000000		0.5000000000000		0.0019304541362
0.5000000000000		0.5000000000000		0.0000000008425
\end{verbatim}
}

\subsection{Two dimensional gnuplot (including time)}

These options produce data suitable for using with the gnuplot visualization client in the format
{\tt
\begin{verbatim}
t x y f(t,x,y,z=fixed)
\end{verbatim}
}



{\tt
\begin{verbatim}
IOASCII::out2D_every = 1
IOASCII::out2D_vars = "wavetoy::phi"
IOASCII::out2D_style = "gnuplot f(t,x,y)"
\end{verbatim}
}

\noindent
{\bf Output file: phi\_2d\_xy.gnuplot}
{\tt
\begin{verbatim}
#Parameter file wavetoy.par
#Created Sun 19 Aug 2001 16:33:07
#x-label x
#y-label y
#z-label WAVETOY::phi (z = 0.0000000000000), (zi = 1)


#Time = 0.0000000000000
0.0000000000000		-0.5000000000000		-0.5000000000000		0.0000000000000
0.0000000000000		0.0000000000000		-0.5000000000000		0.0000000000139
0.0000000000000		0.5000000000000		-0.5000000000000		0.0000000000000

0.0000000000000		-0.5000000000000		0.0000000000000		0.0000000000139
0.0000000000000		0.0000000000000		0.0000000000000		1.0000000000000
0.0000000000000		0.5000000000000		0.0000000000000		0.0000000000139

0.0000000000000		-0.5000000000000		0.5000000000000		0.0000000000000
0.0000000000000		0.0000000000000		0.5000000000000		0.0000000000139
0.0000000000000		0.5000000000000		0.5000000000000		0.0000000000000



#Time = 0.2500000000000
0.2500000000000		-0.5000000000000		-0.5000000000000		0.0000000000000
0.2500000000000		0.0000000000000		-0.5000000000000		0.0000000000000
0.2500000000000		0.5000000000000		-0.5000000000000		0.0000000000000

0.2500000000000		-0.5000000000000		0.0000000000000		0.0000000000000
0.2500000000000		0.0000000000000		0.0000000000000		0.4980695458846
0.2500000000000		0.5000000000000		0.0000000000000		0.0000000000000

0.2500000000000		-0.5000000000000		0.5000000000000		0.0000000000000
0.2500000000000		0.0000000000000		0.5000000000000		0.0000000000000
0.2500000000000		0.5000000000000		0.5000000000000		0.0000000000000



#Time = 0.5000000000000
0.5000000000000		-0.5000000000000		-0.5000000000000		0.0000000008425
0.5000000000000		0.0000000000000		-0.5000000000000		0.0019304541362
0.5000000000000		0.5000000000000		-0.5000000000000		0.0000000008425

0.5000000000000		-0.5000000000000		0.0000000000000		0.0019304541362
0.5000000000000		0.0000000000000		0.0000000000000		-0.7509652270577
0.5000000000000		0.5000000000000		0.0000000000000		0.0019304541362

0.5000000000000		-0.5000000000000		0.5000000000000		0.0000000008425
0.5000000000000		0.0000000000000		0.5000000000000		0.0019304541362
0.5000000000000		0.5000000000000		0.5000000000000		0.0000000008425
\end{verbatim}
}

%
% Automatically created from the ccl files 
% Do not worry for now.
\include{interface}
\include{param}
\include{schedule}

\end{document}
