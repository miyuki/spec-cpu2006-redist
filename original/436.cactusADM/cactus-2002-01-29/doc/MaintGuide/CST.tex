% /*@@
%   @file      CST.tex
%   @date      Wed Jan 12 14:38:29 2000
%   @author    Tom Goodale
%   @desc 
%   
%   @enddesc 
%   @version $Header: /cactus/Cactus/doc/MaintGuide/CST.tex,v 1.2 2001/05/11 10:40:34 goodale Exp $
% @@*/

\begin{cactuspart}{4}{The CST}{$RCSfile: CST.tex,v $}{$Revision: 1.2 $}
\renewcommand{\thepage}{\Alph{part}\arabic{page}}

\chapter{Introduction}

The CST is really the glue which holds the code together.  It takes the 
specifications which users have provided in their {\em .ccl} files
and generates C header and source files which are used to tell the 
flesh about the thorns.

The processing is done in three stages.  In the first stage the {\em .ccl}
files from each thorn in the {\em ThornList} are parsed and the data from them
is stored internally in databases.  In the second stage the data is 
cross-indexed for consistency.  Finally the files are written out into the 
{\em bindings} directory in the configuration directory.

\chapter{The Databases}

\chapter{The Generated Files}

\chapter{The Parsing Routines}

\chapter{The Output Routines}

\chapter{Miscellaneous Routines}

%%%%%%%%%%%%%%%%%%%%%%%%%%%%%%%%%%%%%%%%%%%%%%%%%%%%%%%%%%%%%%%%%%%%%%%%%%
\end{cactuspart}
