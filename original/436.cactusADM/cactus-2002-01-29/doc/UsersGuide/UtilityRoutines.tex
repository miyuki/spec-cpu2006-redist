% /*@@
%   @file      UtilityRoutines.tex
%   @date      27 Jan 1999
%   @author    Jonathan Thornburg
%   @desc 
%   Description of the Util_* utility routines for the Cactus User's Guide.
%   @enddesc 
%   @version $Header: /cactus/Cactus/doc/UsersGuide/UtilityRoutines.tex,v 1.6 2002/01/21 16:04:49 jthorn Exp $
% @@*/
\begin{cactuspart}{3}{Utility Routines}{$RCSfile: UtilityRoutines.tex,v $}{$Revision: 1.6 $}
\renewcommand{\thepage}{\Alph{part}\arabic{page}}

%%%%%%%%%%%%%%%%%%%%%%%%%%%%%%%%%%%%%%%%%%%%%%%%%%%%%%%%%%%%%%%%%%%%%%%%%%%%%%%%

\chapter{Introduction} 

As well as the high-level \verb|CCTK_|* routines, Cactus also
provides a set of lower-level \verb|Util_|* utility routines, which
are mostly independent of the rest of Cactus.  This chapter gives a
general overview of programming with these utility routines.

%%%%%%%%%%%%%%%%%%%%%%%%%%%%%%%%%%%%%%%%%%%%%%%%%%%%%%%%%%%%%%%%%%%%%%%%%%%%%%%%

\chapter{Key/Value Tables}

%%%%%%%%%%%%%%%%%%%%%%%%%%%%%%%%%%%%%%%%

\section{Motivation}

Various Cactus functions need to pass information through a generic
interface.  In the past we have used various ad-hoc means to do this,
and we often had trouble passing "extra" information that wasn't
anticipated in the original design.  For example, for interpolation
\verb|CCTK_InterpLocal()| and \verb|CCTK_InterpGV()| require that
any parameters (such as the order of the interpolant) be encoded
into the interpolator's character string name.  Similarly, elliptic
solvers often need to pass various parameters, but we don't have a
good way to do this.

Key/value tables (``tables'' for short) provide a clean solution
to these problems.  They're implemented by the \verb|Util_Table|*
functions (described in detail in
section~\ref{sect-FunctionReference/UtilityFunctions}).

%%%%%%%%%%%%%%%%%%%%%%%%%%%%%%%%%%%%%%%%

\section{The Basic Idea}

Basically, a table is an object which maps strings to almost-arbitrary
user-defined data.  (If you know Perl, a table is very much like a
Perl hash table.)%%%
\footnote{%%%
	 However, the present Cactus tables implementation
	 is optimized for a relatively small number of
	 distinct keys in any one table.  It will still
	 work ok for huge numbers of keys, but it will be
	 slow.
	 }%%%

More formally, a table is an object which stores a set of ``keys''
and a corresponding set of ``values''.

Keys are C-style null-terminated character strings, with the slash
character \verb|'/'| reserved for future expansion.%%%
\footnote{%%%
	 Think of hierarchical tables for storing
	 tree-like data structures.%%%
	 }%%%

Values are 1-dimensional arrays of any of the usual Cactus data types
described in section~\ref{sect-ThornWriting/DataTypes}.
A string can be stored by treating it as a 1-dimensional array of
\verb|CCTK_CHAR| (there's an example of this below).

The basic ``life cycle'' of a table is that you first create it
with \verb|Util_TableCreate()|, then set values in it with one or
more of the \verb|Util_TableSet*()| and/or \verb|Util_TableSet*Array()|
functions, then later on some other piece of code can get (copies of)
the values you set from the table with one or more of the
\verb|Util_TableGet*()| and/or \verb|Util_TableGet*Array()| functions.
Finally, when you're through with a table you destroy it with
\verb|Util_TableDestroy()|.

For simple uses, there's also a short-cut function
\verb|Util_TableCreateFromString()|, which creates a table and
sets some integer or real values in it, based on a parameter-file--style
string.

%%%%%%%%%%%%%%%%%%%%%%%%%%%%%%%%%%%%%%%%

\section{A Simple Example}

Here's a simple example (in C)%%%
\footnote{%%%
	 In theory tables should be usable from both
	 C and Fortran, but noone has written the Fortran
	 wrappers yet, so at present tables are C-only.
	 }%%%
{} of how to use a table:
\begin{verbatim}
#include "util_Table.h"
#include "cctk.h"

/* create a table and set some values in it */
int handle = Util_TableCreate(UTIL_TABLE_FLAGS_DEFAULT);
Util_TableSetInt(handle, 2, "two");
Util_TableSetReal(handle, 3.14, "pi");

...

/* get the values from the table */
CCTK_INT two_value;
CCTK_REAL pi_value;
Util_TableGetInt(handle, &two_value, "two");    /* sets two_value = 2 */
Util_TableGetReal(handle, &pi_value, "pi");     /* sets pi_value = 3.14 */
\end{verbatim}

Actually, you shouldn't write code like this -- in the real world
errors sometimes happen, and it's not at all nice to have your program
crash.  The {\em right\/} thing to do is to always check for errors.
To allow this, all the table routines return a status, which is
zero or positive for a successful return, but negative if and only
if some sort of error has occured.%%%
\footnote{%%%
	 Often (as in the examples here) you don't care
	 about the details of which error occured.  But if
	 you do, there are various error codes defined in
	 {\t "util\_Table.h"} and {\t "util\_ErrorCodes.h"};
	 the detailed function descriptions in
	 section~\ref{sect-FunctionReference/UtilityFunctions}
	 say which error codes each function can return.
	 }%%%
{}  So, the above example can (should) be rewritten like this:

\begin{verbatim}
#include <util_Table.h>

/* create a table and set some values in it */
int handle = Util_TableCreate(UTIL_TABLE_FLAGS_DEFAULT);
if (handle < 0)
        CCTK_WARN(-1, "couldn't create table!");

/* try to set two_value = 2 and set pi_value = 3.14 */
if (Util_TableSetInt(handle, 2, "two") < 0)
        CCTK_WARN(-1, "couldn't set integer value in table!");
if (Util_TableSetReal(handle, 3.14, "pi") < 0)
        CCTK_WARN(-1, "couldn't set real value in table!");

...

/* get the values from the table */
CCTK_INT two_value;
CCTK_REAL pi_value;
/* try to get two_value and pi_value from the table */
if (Util_TableGetInt(handle, &two_value, "two") < 0)
        CCTK_WARN(-1, "couldn't get integer value from table!");
if (Util_TableGetReal(handle, &pi_value, "pi") < 0)
        CCTK_WARN(-1, "couldn't get integer value from table!");

/* now two_value = 2 and pi_value = 3.14 */
\end{verbatim}

%%%%%%%%%%%%%%%%%%%%%%%%%%%%%%%%%%%%%%%%

\section{Convenience Routines}

There are also convenience routines for the common case of setting
values in a table based on a string.

For example, the following code sets up exactly the same table as the
previous example:

\begin{verbatim}
#include <util_Table.h>

/* create a table and set some values in it */
int handle = Util_TableCreate(UTIL_TABLE_FLAGS_DEFAULT);
if (handle < 0)
        CCTK_WARN(-1, "couldn't create table!");

/* try to set two_value = 2 and set pi_value = 3.14 */
if (Util_TableSetFromString(handle, "two=2 pi=3.14") != 2)
        CCTK_WARN(-1, "couldn't set values in table!");
\end{verbatim}

There is also an even higher-level convenience function
\verb|Util_TableCreateFromString()|: this creates a table with the
case-insensitive flag set (to match Cactus parameter-file semantics),
then (assuming no errors occured) calls \verb|Util_TableSetFromString()|
to set values in the table.  For example, apart from the case-insensitive
flag, the following code sets up exactly the same table as the
previous two examples:

\begin{verbatim}
#include <util_Table.h>

int handle = Util_TableCreateFromString("two=2 pi=3.14");
if (handle < 0)
        CCTK_WARN(-1, "couldn't create table from string!");
\end{verbatim}

%%%%%%%%%%%%%%%%%%%%%%%%%%%%%%%%%%%%%%%%

\section{Arrays as Table Values}

As well as a single numbers (or characters or pointers), tables can
also store 1-dimensional arrays of numbers (or characters or pointers).%%%
\footnote{%%%
	 The table makes (stores) a {\em copy\/} of the array
	 you pass in, so you probably shouldn't use tables to
	 store really big arrays like grid functions.  But it's
	 fine to store things like parameters and coefficients.
	 (If you do have a really big array, consider storing
	 a pointer to it.)%%%
	 }%%%

For example (continuing the previous example):
\begin{verbatim}
static const CCTK_INT a[3] = { 42, 69, 105 };
if (Util_TableSetIntArray(handle, 3, a, "my array") < 0)
        CCTK_WARN(-1, "couldn't set integer array value in table!");

...

CCTK_INT blah[10];
int count = Util_TableGetIntArray(handle, 10, blah, "my array");
if (count < 0)
        CCTK_WARN(-1, "couldn't get integer array value from table!");
/* now count = 3, blah[0] = 42, blah[1] = 69, blah[2] = 105, */
/*     and all remaining elements of blah[] are unchanged */
\end{verbatim}
As you can see, a table entry remembers the length of any array
value that has been stored in it.%%%
\footnote{%%%
	 In fact, actually {\em all\/} table values are
	 arrays -- setting or getting a single value is
	 just the special case where the array length is 1.
	 }%%%
{}

If you only want the first few values of a larger array, just pass
in the appropriate length of your array:
that's ok:
\begin{verbatim}
CCTK_INT blah2[2];
int count = Util_TableGetIntArray(handle, 2, blah2, "my array");
if (count < 0)
        CCTK_WARN(-1, "couldn't get integer array value from table!");
/* now count = 3, blah2[0] = 42, blah2[1] = 69 */
\end{verbatim}
You can even ask for just the first value:
\begin{verbatim}
CCTK_INT blah1;
int count = Util_TableGetInt(handle, blah1, "my array");
if (count < 0)
        CCTK_WARN(-1, "couldn't get integer array value from table!");
/* now count = 3, blah1 = 42 */
\end{verbatim}

%%%%%%%%%%%%%%%%%%%%%%%%%%%%%%%%%%%%%%%%

\section{Character Strings}

One very common thing you might want to store in a table is a
character string.  While you could do this by explicitly storing
an array of \verb|CCTK_CHAR|, there are also more-convenient routines
specially for setting and getting strings:
\begin{verbatim}
if (Util_TableSetString(handle, "black holes are fun", "bh") < 0)
        CCTK_WARN(-1, "couldn't set string value in table!");

...
char buffer[50];
if (Util_TableGetString(handle, 50, buffer, "bh") < 0)
        CCTK_WARN(-1, "couldn't get string value from table!");

/* now buffer[] contains the string "black holes are fun" */
\end{verbatim}

\verb|Util_TableGetString()| guarantees that the string is
terminated by a null character (\verb|'\0'|), and also returns an
error if the string is too long for the buffer.

%%%%%%%%%%%%%%%%%%%%%%%%%%%%%%%%%%%%%%%%

\section{Table Options}

A table has an integer ``flags word'' which may be used to specify
various options, via bit flags defined in \verb|"util_Table.h"|.
At present there is only a single option:
\begin{description}
\item[Keys May be Case-Sensitive or Case-Insensitive:]
	By default keys are treated as C-style character strings,
	and the table functions compare them with the standard C
	\verb|strcmp()| function.
	However, by setting the \verb|UTIL_TABLE_FLAGS_CASE_INSENSITIVE|
	bit in the flags word, this table's keys may be made
	case-insensitive, \ie{} the table routines then compare
	this table's keys with \verb|Util_StrCmpi()|.%%%
\footnote{%%%
 	 It's not entirely clear how non-ASCII characters
	 like \c{c}, \'{e}, \ss, \dots should be handled in
	 this case (some of them have only a single case).
	 And what about the 16-bit Asian character sets
	 used for Chinese, Japanese, Korean, \dots.
	 In general, Cactus basically just ignores
	 internationalization issues.  Sigh\dots
	 }%%%
{}
	Note that keys are still {\em stored\/} exactly as
	the caller specifies them (\ie{} they are {\em not\/}
	forced into a canonical case); it's only their
	{\em comparison\/} that's affected by this flag.
\end{description}

%%%%%%%%%%%%%%%%%%%%%%%%%%%%%%%%%%%%%%%%

\section{Table Iterators}

In the examples up to now, the code which wanted to get values from
the table knew what the keys were.  It's also useful to be able to
write generic code which can operate on a table without knowing the
keys.  ``Table iterators'' (``iterators'' for short) are used for this.

An iterator is an abstraction of a pointer to a particular table entry.
Iterators are to the \verb|DIR *| pointers used by the POSIX
\verb|opendir()|, \verb|readdir()|, \verb|closedir()|, and similar
functions, to Perl hash tables' \verb|each()|, \verb|keys()|,
and \verb|values|, and to the C++ Standard Template Library's
forward iterators.

At any time the entries in a table may be considered to be in some
arbitrary (implementation-defined) order; an iterator may be used to
walk through some or all of the table entries in this order.  This
order is guaranteed to remain unchanged for any given table so long
as no changes are made to that table, \ie{} so long as no
\verb|Util_TableSet*()|, \verb|Util_TableSet*Array()|,
\verb|Util_TableSetString()|, or \verb|Util_TableDeleteKey()| calls
are made on that table (making such calls on other tables doesn't matter).
The order may change if there is any change in the table, and it
may differ even between different tables with identical key/value
contents.%%%
\footnote{%%%
	 For example, if tables were implemented by hashing,
	 the internal order could be that of the hash buckets,
	 and the hash function could depend on the internal
	 table address.
	 }%%%
{}  

Any change in the table also invalidates all iterators pointing
anywhere in the table; using any such iterator is an error.
Multiple iterators may point into the same table; they all use the
same order, and (unlike in Perl) they're all independent.

The detailed function description
(section~\ref{sect-FunctionReference/UtilityFunctions})
for \verb|Util_TableItQueryKeyValueInfo()| has an example of
using an iterator to print out all the entries in a table.

%%%%%%%%%%%%%%%%%%%%%%%%%%%%%%%%%%%%%%%%

\section{Multithreading and Multiprocessor Issues}

{\bf At the moment, the table functions are {\em not\/} thread-safe
in a multithreaded environment!}  However, this should change in
the not-too-distant future: then all the table functions will default
to being thread-safe.  That is, user code will be able call these
functions concurrently from multiple threads, with the table functions
automatically doing any necessary locking.%%%
\footnote{%%%
	 For the implementation, this means that we will need a
	 reader/writer lock for each table and for each iterator:
	 any number of threads will be able to concurrently read
	 the data structure, but any write will require exclusive
	 access.
	 }%%%
{}  (We may add a flags-word bit to suppress this for maximum
performance if you know you won't be making concurrent calls from
multiple threads.)

Note that tables and iterators will still be process-wide, i.e. all
threads see the same tables and iterators (think of them as like the
Unix current working directory, with the various routines which modify
the table or change iterators acting like a Unix \verb|chdir()| system
call).

In a multiprocessor environment, tables are always processor-local.

%%%%%%%%%%%%%%%%%%%%%%%%%%%%%%%%%%%%%%%%

\section{Metadata about All Tables}

We have decided that tables do not {\em themselves\/} have names or other
attributes.  However, at some time in the future we may add some special
``system tables'' by Cactus itself to store this sort of information
for those cases where it's needed.

For example, at some time in the future we may add support for a
``checkpoint me'' bit in a table's flags word, so that if you want a
table to be checkpointed, you just need to set this bit.
In this case the table will probably get a system-generated name in
the checkpoint dump file.  But if you want the table to have some
other name in the dump file, then you need to tell the checkpointing
code that by setting an appropriate entry in a checkpoint table.
(You would find the checkpoing table by looking in a special
``master system table'' that records handles of other interesting tables.)

%%%%%%%%%%%%%%%%%%%%%%%%%%%%%%%%%%%%%%%%%%%%%%%%%%%%%%%%%%%%%%%%%%%%%%%%%%%%%%%%

\end{cactuspart}
