% /*@@
%   @file      Appendices.tex
%   @date      27 Jan 1999
%   @author    Tom Goodale, Gabrielle Allen, Gerd Lanferman
%   @desc
%   Appendices for the Cactus User's Guide
%   @enddesc
%   @version $Header: /cactus/Cactus/doc/UsersGuide/Appendices.tex,v 1.31 2002/01/19 16:34:07 tradke Exp $
% @@*/
\begin{cactuspart}{6}{Appendices}{$RCSfile: Appendices.tex,v $}{$Revision: 1.31 $}
\renewcommand{\thepage}{\Alph{part}\arabic{page}}

\chapter{Glossary}
\label{sec:glossary}

\begin{Lentry}

\item[{\tt arrangement}] A collection of thorns.
\item[{\tt autoconf}] A GNU program which builds a configuration
   script which can be used to make a Makefile.
\item[{\tt Cactus}] A green fleshy plant with lots of thorns, usually painful if touched.
\item[{\tt CCTK}] Cactus Computational Tool Kit (The Cactus flesh and
            computational thorns).
\item[{\tt CCL}] The {\tt Cactus Configuration Language}, this is the language
that the thorn configuration files are written in.
\item[{\tt configuration}]
\item[{\tt checkout}] Get a copy of the code or thorns from CVS.
\item[{\tt checkpointing}] Save the entire state of a run to file so that it can be restarted at a later time.
\item[{\tt convergence}]
\item[{\tt CST}] This stands for Cactus Specification Tool, which is
the perl scripts which parse the thorns' configuration files (those
that have a \texttt{.ccl} extension).
\item[{\tt CVS}] The {\em "Concurrent Versioning System"} is the
  favoured distribution system for Cactus and can be
  downloaded from your favorite GNU site.
\item[{\tt driver}] A thorn which creates and handles grid hierachies.
\item[{\tt flesh}] The routines which hold all the thorns together, this
is what you get if you check out {\tt Cactus} from our CVS repository.
\item[{\tt friend}]
\item[{\tt GF}] Shorthand for a {\tt grid function}.
\item[{\tt gmake}]
\item[{\tt ghostzone}]
\item[{\tt grid function}]
\item[{\tt grid hierarchy}] A computational grid, and the grid variables associated with it.
\item[{\tt grid scalar}]
\item[{\tt grid variable}]
\item[{\tt GNATS}] The GNU program we use for reporting and tracking bugs,
  comments and suggestions.
\item[{\tt handle}] A signed integer value $>= 0$. Many Cactus routines return
  or accept a handle to represent a dynamic data or code object. Handles for
  the same object class should not be trusted to comprise a consecutive
  sequence of integer values.
\item[{\tt HDF5}] Hierarchical Data Format version 5.
\item[{\tt implementation}]
\item[{\tt inherit}]
\item[{\tt interpolation}]
\item[{\tt MPI}]
\item[{\tt parameter}] A variable which remains unchanged throughout the execution of a Cactus executable. Parameters all have default values which can be changed in a parameter file.
\item[{\tt processor topology}]
\item[{\tt PUGH}] The default parallel driver for Cactus which uses MPI.
\item[{\tt PVM}]  {\tt Parallel Virtual Machine}, provides interprocessor
  communication.
\item[{\tt reduction}]
\item[{\tt TAGS}]
\item[{\tt target}]
\item[{\tt thorn}] A collection of subroutines with a definite interface
             and purpose.
\item[{\tt WMPI}] Win32 Message Passing Interface.

\end{Lentry}


\chapter{Configuration file syntax}
\label{sec:cofisy}

\section{General Concepts}
\label{sec:geco}

Each thorn is configured by three compulsory files in the top level thorn
directory:
\begin{itemize}
\item{} {\tt interface.ccl}
\item{} {\tt param.ccl}
\item{} {\tt schedule.ccl}
\end{itemize}
These files are written in the {\it Cactus Configuration Language} which is
case insensitive.

\section{interface.ccl}
\label{sec:in}

The interface configuration file consists of
\begin{itemize}
\item a header block giving details
of the thorns relationship with other thorns
\item a series of blocks
listing the thorn's global variables.
\item details of other interactions with thorns:
\begin{itemize}
\item building include files
\end{itemize}
\end{itemize}

The header block has the form:
{\t
\begin{verbatim}
implements: <implementation>
[inherits: <implementation>, <implementation>]
[friend: <implementation>, <implementation>]
\end{verbatim}
}
where
\begin{itemize}
\item{} The implementation name must be unique among all thorns, except
        between thorns which have the same public and protected variables and
        global and restricted parameters.
\item{} Inheriting from another implementation makes all that implementation's
        public variables available to your thorn. At least one thorn
        providing any inherited implementation must be present at compile time.
        A thorn cannot inherit from itself. Inheritance is associative and
        recursive, but not commutative.
\item{} Being a friend of another implementation makes all that
        implementations
        protected variables available to your thorn. At least one thorn
        providing an implementation for each friend must be present at
        compile time. A thorn cannot be its own friend. Friendship is
        associative and commutative. Note that your thorn is also a friend
        of all your thorns friend's friends.
\end{itemize}

The thorn's variables are collected into groups. This is not only
for convenience, but for collecting like variables together.
Storage assignment, communication assignment and ghostzone synchronization
takes place for groups only.


The thorn's variables are defined by:
{\t
\begin{verbatim}
[<access>:]

<data_type> <group_name> [TYPE=<group_type>] [DIM=<dim>] [TIMELEVELS=<num>] [SIZE=<size in each direction>] [DISTRIB=<distribution_type>] [GHOSTSIZE=<ghostsize>] [STAGGER=<stagger-specification>]
[{
   <variable_name>[,]<variable_name>
   <variable_name>
}] ["<group_description>"]
\end{verbatim}}

\begin{itemize}
\item{} {\t access} defines which thorns can use the following
        groups of variables. {\t access} can be either
        {\t public}, {\t protected} or {\t private}.
\item{} {\t data\_type} defines the data type of the variables in the group.
Supported data types are {\t BOOLEAN}, {\t INTEGER}, {\t CHAR} and {\t REAL}.
(In the future {\t COMPLEX} will also be supported.)
\item{} {\t group\_name} must be a alpha-numeric name (which may also
contain underscores) which is unique
within the scope of the thorn. A group name is compulsory.
\item{} {\t TYPE} designates the kind of variables held by the group.
The choices are {\t GF}, {\t ARRAY} or {\t SCALAR}. This field is
optional, with the default variable type being {\t SCALAR}.
\item{} {\t DIM} defines the spatial dimension if the group is
	of type {\t ARRAY} or {\t GF}, and can take the value
	{\t 1}, {\t 2}, or {\t 3}.
The default value is {\t DIM=3}.
\item{} {\t TIMELEVELS} defines the number of timelevels a group has if
        the group is of type {\t ARRAY} or {\t GF}, and can take any positive
        value.
\item{} {\t SIZE} defines the number grid-points an {\tt ARRAY} has in each direction.
        This should be a comma-separated list of positive integer constants or parameter names (optionally with an integer constant added/substracted to/from it).
\item{} {\t DISTRIB} defines the processor decomposition of an {\tt ARRAY}.
        {\tt DISTRIB=CONSTANT} implies that {\tt SIZE} grid-points should
        be allocated on each processor.
\item{} {\t GHOSTSIZE} defines number of ghost-zones in each dimension of an {\tt ARRAY}.
\item{} {\t STAGGER} defines position of grid-points of a {\tt GF} with respect to
        the underlying grid.  It consists of a string made up of a combination {\tt DIM}
        of the letters {\tt M}, {\tt C}, {\tt P} depending on whether the layout in
        that direction is on the {\tt M}inus face, {\tt C}entre, or {\tt P}lus face
        of the cell in that dimension.
\item{} The (optional) block following the group declaration line contains a list of
        variables contained in the group. All variables in a group have
        the same data type, variable type and dimension. The list can be
        separated by spaces, commas, or new lines. The variable names
        must be unique within the scope of the thorn.
        A variable can only be a member of
        one group. The block must be delimited by brackets on new lines.
        If no block is given after a group declaration line, a
        variable with the same name as the group is created.
\item{} An optional description of the group can be given on the last line,
        at the moment this description is not used for any purpose.
\end{itemize}

\section{param.ccl}
\label{sec:pa}

The parameter configuration file consists of a list of
{\it parameter object specification items} (OSIs) giving the type and
range of the parameter separated by optional
{\it parameter data scoping items} (DSIs) which detail access to the
parameter.

\subsection{Parameter data scoping items}
{\tt
\begin{verbatim}
<access>:
\end{verbatim}
}
The key word {\t access} designates that all parameter object specification
items up to the next parameter data scoping item are in the same
protection or scoping class. {\tt access} can take the values:
\begin{Lentry}
\item[{\tt global}] all thorns have access to global parameters
\item[{\tt restricted}] other thorns can have access to these
                           parameters if they specifically request
                           it in their own param.ccl
\item[{\tt private}] only your thorn has access to private parameters
\item[{\tt shares}] in this case an {\t implementation} name must follow
                        the colon. It declare that all the parameters in
                        the following scoping block are restricted variables
                        from the specified {\tt implementation}.
\end{Lentry}


\subsection{Parameter object specification items}
\label{sec:paobspit}

{\t
\begin{verbatim}
[EXTENDS|USES] <parameter_type> <parameter name> "<parameter description>"
{
  <PARAMETER_VALUES>
} <default value>
\end{verbatim}
}
\begin{itemize}
\item{} Allowed {\t parameter\_type}s are
  \begin{Lentry}
     \item[{\t INTEGER}] The specification of {\t parameter\_value}s takes
     the form of any number of comma-separated blocks of the form:
{\t
\begin{verbatim}
[<low-range>][:[<high-range>][:[<step>]]][::"<range description>"]
\end{verbatim}
}
Where an empty field, or a {\t *} in the place of {\tt low-range} or
{\t high-range} indicates $-\infty$ and $\infty$ respectively. The
default value for {\t step} is 1. A number by itself denotes that
this number is the only acceptable value.
     \item[{\t REAL}] The range specification is the same as integers,
             except that here, no {\t step} implies a continuum of values.
     \item[{\t KEYWORD}] Each entry in the list of acceptable values
             for a keyword has the form
{\t
\begin{verbatim}
"<keyword value>", "<keyword value>" :: "<description>"
\end{verbatim}
}
     \item[{\t STRING}]  Allowed values for strings should be specified using regular expressions. To allow any string, the regular expression ".*" should be used.
     \item[{\t BOOLEAN}] No {\t PARAMETER\_VALUES} should be specified for a boolean
             parameter. The default value for a boolean can be
        \begin{itemize}
           \item{} True: {\t 1}, {\t yes}, {\t y}, {\t t}, {\t true}
           \item{} False: {\t 0}, {\t no}, {\t n}, {\t f}, {\t false}
        \end{itemize}
  \end{Lentry}

\item{} The {\t parameter name} must be unique within the scope of the thorn.

\item{} A thorn can declare that it {\t EXTENDS} a parameter that it is a
        friend of. This allows it to declare additional acceptable values.
        By default it is acceptable for two thorns to declare the same
        value as acceptable.
\end{itemize}


\section{schedule.ccl}
\label{sec:sc}

The schedule configuration files consists of
\begin{itemize}
\item{} assignments to switch on storage and communications
        for array variables at the start of program execution.
\item{} {\it schedule blocks} which schedule a subroutine from the
        thorn to be called at a specified time during program
        execution. Statements within the schedule block can
        be used to switch on storage and communication for groups
        of variables during the duration that the subroutine is called.
\item{} Conditional statements
\end{itemize}

{\it assignments statements} have the form:
{\t
\begin{verbatim}
[STORAGE: <group>, <group>]
[COMMUNICATION: <group>, <group>]
\end{verbatim}
}

Each {\it schedule block} in the file {\t schedule.ccl} must have
the syntax:
{\t
\begin{verbatim}
schedule [GROUP] <function name> AT|IN <time> [BEFORE|AFTER <group>] [WHILE <variable>] [AS <alias>]
{
  LANG: <language>
  [STORAGE:       <group>,<group>...]
  [TRIGGER:       <group>,<group>...]
  [SYNCHRONISE:   <group>,<group>...]
  [OPTIONS:       <option>,<option>...]
} "Description of function"
\end{verbatim}
}

The keywords {\t STORAGE} and {\t COMMUNICATION} as well as the options keywords
inside of a schedule block are significant with their first four letters.

Any schedule block or assignment statements can be optionally
surrounded by conditional {\t if-elseif-else}
constructs using the parameter data base. These can be nested,
and have the general form:
{\t
\begin{verbatim}
if (CCTK_Equals(<parameter>,<string>))
{
  [<assignments>]
  [<schedule blocks>]
}
\end{verbatim}
}

Conditional constructs cannot be used inside of a schedule block.

{\tt STORAGE} may also appear as a statement outside a schedule block,
in which case it switches storage on for the specified groups permanently,
subject to any conditional statements around the {\tt STORAGE} statement.

\subsection{Allowed Options}

\begin{Lentry}

\item[{\tt GLOBAL}]
This routine will only be called once on a grid hierarchy, not for all
subgrids making up the hierarchy.  This can be used, for example, for
analysis routines which use global reduction or interpolation routines
rather than the local subgrid passed to them, and hence should only be
called once.

\end{Lentry}

\chapter{Flesh parameters}
\label{sec:ccpa}

The flesh parameters are defined in the file {\tt src/par/param.ccl}.

\section{Private parameters}

\begin{Lentry}



\item[{\tt cctk\_timer\_output}]
Give timing information [{\tt yes}] \{{\tt off}, {\tt full}\}

\item[{\tt cctk\_full\_warnings}]
Give detailed information for each warning statement [{\tt yes}]

\item[{\tt cctk\_strong\_param\_check}]
Die on parameter errors in {\tt CCTK\_PARAMCHECK} [{\tt yes}]

\item [{\tt cctk\_show\_schedule}]
Print the scheduling tree to standard output [{\tt yes}]

\item [{\tt cctk\_show\_banners}]
Show any registered banners for the different thorns [{\tt yes}]

\item [{\tt cctk\_brief\_output}]
Give only brief output [{\tt no}]

\end{Lentry}

\section{Restricted parameters}

\begin{Lentry}

\item[{\tt cctk\_initial\_time}]
Initial time for evolution [{\tt 0.0}]

\item [{\tt cctk\_final\_time}] Final time for evolution, overriden by
{\tt cctk\_itlast} unless it is positive [{\tt -1.0}]

\item [{\tt cctk\_itlast}]
Final iteration number [{\tt 10}]

\end{Lentry}

\chapter{Using {\tt GNATS}}
GNATS is a freely redistributable set of tools for tracking bug
reports. It allows users to categorize their problem report and submit
them to the GNATS. The bugtracker will assign appropriate maintainers
to analyze and solve the problem.
We are currently supporting a web based interface at {\tt
http://www.cactuscode.org} which lets you
interactively file a bug report. Here, we briefly
describe the main categories when creating a Cactus
problem report.
\begin{Lentry}
\item[{\bf Reporters email}] Your email address so we can get in
contact with you.
\item[{\bf Category}] there is currently a category for each of the
Cactus thorns and arrangements, also a category for the old Cactus3.x and
some general subjects like Web,etc. Select whatever category fits
best.
\item[{\bf Synopsis}] A brief and informative subject line.

\item[{\bf Confidential}] Unused, all PRs are public.

\item[{\bf Severity}] Pick one three levels.

\item[{\bf Class}] In the selected category, specify what kind of
problem you are dealing with.

\item[{\bf submitter ID}] Unused

\item[{\bf Originator}] Your name. Anonymous is OK, but you real name
   would be best.

\item[{\bf Release}] The Cactus release you are using. You can find this
   out, for example, from an executable by typing {\tt cactus\_<config> -v}.

\item[{\bf Environment}] Very important: specify the environment,
e.g. {\tt uname -a}, {\tt MPI}/non-{\tt MPI}, etc.

\item[{\bf Description}] Describe your problem precisely, if you get a
core dump, include the stack trace, give the minimal number of thorns,
this problems occurs with.

\item[{\bf How-To-Repeat}] Tell us how to repeat the problem if it is
software related.

\item[{\bf Fix}] If you can provide a fix, let us know.
\end{Lentry}

We also provide the customized {\tt send-pr} and {\tt send-pr.el} programs at
our website. These commands are compiled to submit Cactus problem
reports in your shell and from within emacs, respectively.


\label{sec:usgn}

\chapter{Using {\tt CVS}}
\label{sec:uscv}
{\tt CVS} is a version control system, which allows you to  keep
old versions of files (usually source code), log of
when, and why changes occurred, and who made them,  etc.
Unlike the simpler systems, {\tt CVS} does not just operate on one file at a
time or one directory at a time,  but
operates  on  hierarchical collections of directories consisting of
version controlled files.  {\tt CVS} helps to  manage
releases  and  to control the concurrent editing of source
files among multiple  authors. {\tt CVS} can be obtained from
{\tt http://www.cyclic.com}.

A {\em CVS repository} located on a {\em server} may consist of an arbitrary
number of {\em modules}, which can be checked out (that is downloaded)
independently. The Cactus flesh and the Cactus
arrangements are organized as modules, their {\em CVS server} is {\tt cvs.cactuscode.org}.

\section{Essential CVS commands}

\begin{Lentry}
\item[{\bf cvs login}]
Logs into the repository. You will be prompted for a {\em
password}. This cvs command leaves a file {\tt .cvspass} in your
home directory. There is no need to login every time you issue a cvs
command, as long as this file exists. For a Cactus checkout, you have
to log into the CVS server, using the cvs option {\bf -d} to specify CVSROOT:\\
{\tt cvs -d :pserver:cvs\_anon@cvs.cactuscode.org:/cactus login}

{\tt Note}: that there is no "logout" command: if you log in with
administrative rights from an arbitrary account, you should be aware
that the password file enables subsequent administrative logins from
that account. {\em Delete the file if necessary}.

\item[{\bf cvs checkout} {\em modules} \ldots]
This command  creates
your private copy of the source for {\em modules}. You can work
with this copy  without  interfering  with  others'
work.   At  least  one subdirectory level is always created: it does
not write all files into your current directory but creates a
directory. For Cactus, you need to either include the {\bf -d} options to
specify the CVSROOT directory and the CVS server, or specify them
with an environment variable (see below). Once you
have checked out the repository there is no need to include the {\bf
-d} option and its rather lengthy argument: the necessary information
is contained in the local {\tt CVS/} directories.

\item[{\bf cvs update}]
Execute  this  command  from  {\it within}  your  private
source  directory  when  you  wish  to  update your
copies of source  files  from  changes  that  other
developers have made to the source in the repository.
Merges are performed automatically when possible, a warning is issued
if manual  resolution  is  required  for conflicting  changes. If your
local copy is several versions behind the actual repository copy, CVS
will {\em refetch} the whole file instead of applying multiple
patches.

\item[{\bf cvs add} {\tt file}]
Use this command to enroll new files in cvs records
of your working directory.  The files will be added
to the  repository  the  next  time  you  run  `cvs
commit'.

\item[{\bf cvs commit} {\tt file}]
Use this command to add your local changes to the source to
the repository and thereby making it publically available to
checkouts and updates by other users. You cannot commit a
newly created file unless you have {\it added} it.

\item[{\bf cvs diff} {\tt file}]
Show differences between a file in your working directory
and  a file in the source repository, or between two revisions in
source repository.  (Does not change either repository or working
directory.) For example, to see the difference between versions
{\tt 1.8} and {\tt 1.9} of a file {\tt foobar.c}:

{\tt
\begin{verbatim}
cvs diff -r 1.8 1.9 foobar.c
\end{verbatim}
}

\item[{\bf cvs remove} {\tt file}]
Remove files from the source repository, pending  a {\tt cvs commit} on
the same files.

\item[{\bf cvs status} [file]]
This command returns the current status of your local copy relative to
the repository: e.g. it indicates local modifications and possible
updates.

\item[{\bf cvs import} {\tt repository tag1 tag2}]
Import adds an entire source distribution (starting from the
directory you issue the command in) to the repository directory.
Use this command to add new arrangements to the Cactus4.0 repository. The
{\tt repository} argument is a directory name (or a path to a
directory) and the CVS root directory for repositories; to obtain this
directory on the CVS server, send a request to {\tt
cactus@cactuscode.org}. {\tt tag1} and {\tt tag2} are two tags (vendor
 and release tags) that have to be supplied. For example, to add {\tt MyThorn}
to the {\tt MyArrangement} arrangement, which may or may not already exist on
the CVS repository

{\tt
\begin{verbatim}
cvs -d :pserver:<name>@cvs.cactuscode.org:/cactus import MyArrangement/MyThorn
start v1
\end{verbatim}
}

After you import a thorn, you should check it out from the repository straight
away, and only edit this version.

\end{Lentry}

\section{CVS Options}
The CVS command line can include the following:
\begin{Lentry}
\item[{\bf cvs options}] which apply to the overall cvs program
\item[{\bf a cvs command}]  which defines a particular task carried out by
cvs
\item[{\bf command options}] to specify certain working modes for the cvs
command.
\item[{\bf command arguments}] to specify which file to act on.
\end{Lentry}

The options must be put {\em relative} to the {\it cvs command} as the
same option name can mean different things: {\em cvs options} go to the
{\em left} of the cvs command, {\em command options} go to the {\em right}
of the cvs command. Here is a list of essential {\em cvs options}

\begin{Lentry}

\item[{\bf -d} {\em cvs\_root\_directory}]
Use {\em cvs\_root\_directory} as the root directory  pathname  of
the  master source repository.  Overrides
the setting of the  {\em CVSROOT}  environment  variable.
This value should be specified as an absolute pathname.
In the Cactus checkout procedure, you specify the Cactus CVS server:\\
{\tt -d :pserver:cvs\_anon@cvs.cactuscode.org:/cactus/}

\item[{\bf -z} {\em compression-level}]
When transferring files across the network use {\tt gzip}
with compression level  {\em compression-level}  to  compress  and
decompress  data as it is transferred.
Requires the presence of the GNU gzip  program  in
the current search path at both ends of the link.

\item[{\bf -n}]
Do not change any file. Attempt to execute the {\em cvs command} but
only to issue reports. Does not remove, update, etc. any files. Very
effective for testing.

\item[{\bf -v}]
Displays version information of the installed CVS.

\item[{\bf -H} {\em cvs-command}]
Displays usage information about the specified cvs command. Without
cvs-command, a list of all available commands is returned.
\end{Lentry}

Here is a list of essential {\em command options} with the
commands they are used with. They go after the cvs command. For a more
complete list of all options, please refer to the manual page.

\begin{Lentry}

\item[{\bf -P}]
Prune  (remove)  directories  that  are empty after being updated, on
{\bf checkout}, or  {\bf update}.   Normally, an  empty  directory
(one that is void of revision controlled files) is  left  alone.
Specifying  {\bf -P} will cause these directories to be silently
removed from  your  checked-out  sources.   This  does  not  remove
the directory from the repository, only from your checked out copy.

\item[{\bf -m} {\em "Text"}]
Specify a logging message explaining changes, etc. on {\bf commit},
{\bf import}. If you do not specify a message, your default editor
is invoked to allow you to enter one.

\item[\bf -d]
Use this option with the {\bf update} command to create any
directories if they are missing from your local copy. This is normally
the case if another user has added files or directories to the
repository. By default the {\bf update} command only acts on files in
your local copy. Note that omitting this option is a frequent cause of
files missing during compilation.  (You can change this
default behavior of cvs by putting a .cvsrc in your home directory
with the contents ``update -d''.)

\end{Lentry}

\section{CVS Examples}
We list some sample CVS commands to treat the most typical Cactus4.0
CVS situations.
\begin{description}
\item\textbf{Logging into the server}\newline
{\tt cvs -d :pserver:cvs\_anon@cvs.cactuscode.org:/cactus
login} \\ You will be asked for the password for user {\em cvs\_anon}, which is {\tt anon}.

\item\textbf{Checking out the code}\newline
{\tt cvs -d :pserver:cvs\_anon@cvs.cactuscode.org:/cactus
checkout Cactus}\\
check out a CVS module named "Cactus", in this case it checks out the
Cactus Computational Toolkit. A directory {\tt ./Cactus} is created if
it doesn't already exist. If you perform a checkout on an already
existing and locally modified copy of the module, CVS will try to merge the files
with your local copy.

\item\textbf{Updating a file or directory}\newline
Assuming that you have a file {\tt ./foobar} in your checked out
copy, you may perform a \\
{\tt cvs status ./foobar}\\
to inform yourself about the necessary updates, etc. To update the
file issue \\
{\tt cvs update ./foobar}\\
If that was file was locally modified, CVS will try to merge the
changes. Manual merging might be necessary and will be indicated by a
CVS warning.

\item\textbf{Updating a directory}\newline
To recursively update the current directory and all subdirectories,
type\\
{\tt cvs update .}\\
To update a directory {\tt ./mysources}, type\\
{\tt cvs update ./path/to/mysources}

\item\textbf{Committing a changed file}\newline
To commit changes you have applied to your local copy, your file must be in
sync with the repository: your changes must be done to the
latest version, otherwise CVS will instruct you to perform an {\bf
update} first. To commit changes made to a file {\tt ./foobar}, type\\
{\tt cvs commit -m "Reason for the change" ./foobar}\\
You may specify several files to commit.

\item\textbf{Adding and committing a new file}\newline
Adding a new file to the repository is a two fold procedure you first
schedule the file for addition, then you commit it:\\
{\tt cvs add ./newfoo}\\
{\tt cvs commit -m "new few message" ./newfoo}

\item\textbf{Creating a new thorn}\newline

To add a new {\bf module} (e.g. an arrangement) to a Cactus repository we
first have to create a directory for you with the right permissions.
Please contact {\tt cactus@cactuscode.org} providing the name of the
requested module, and who should be able to commit changes to the module.

To add the new module, change directory so that you are in the first directory
that you want to commit to the repository. (e.g. if you want to commit
a new arrangement called {\tt MyArrange} then change directory to
{\tt MyArrange}). Then type\\
{\tt cvs -d :pserver:}{\em your\_login}{\tt
@cvs.cactuscode.org:<repository name> } import {\em module\_name} {\tt start V1}\\
(where {\tt start} and {\tt V1} are the vendor and release tags, which you could change to something different).

To add a new {\bf directory} {\tt <new dir>} to an existing module (that you have write permissions for), either add the directory using\\
{\tt cvs add <new dir>}\\
and then recursing down adding all the new files and directories contained
inside, or import the directory by changing directory to sit inside it, and then using\\
{\tt cvs -d :pserver:}{\em your\_login}{\tt
@cvs.cactuscode.org:<repository\_name> } import {\tt <relative name> start V1}\\
Where {\tt <relative name>} means the position of the directory within the module. (For example, if you have a module called {\em AMod} which contains a
directory {\em BMod}, and you want to add {\em CMod} inside {\em BMod}, then change directory to {\em BMod}, and use {\em AMod/BMod} for the {\em relative name}).


\end{description}

\section{Checking out flesh and thorns with CVS}

\begin{Lentry}
\item[{\bf Login}] Prior to any CVS operation, you need to log into the Cactus
  repository. For an anonymous checkout, type:\\
  {\tt
    cvs -d :pserver:cvs\_anon@cvs.cactuscode.org:/cactus login
    }
  You will be prompted for a password which is {\bf anon}.
\item[{\bf Checkout}] To obtain a fresh copy of Cactus, move to a directory
  which does not contain a previously checked out version, and type
  {\t
    cvs -d :pserver:cvs\_anon@cvs.cactuscode.org:/cactus checkout Cactus
    }
  The CVS checkout procedure will create a directory called {\bf
  Cactus} and install the code inside this directory.  From now on we
  will reference all directory names relative to {\bf Cactus}.

\noindent
  If you want to compile Cactus with thorns, you now need to checkout
  separately the required arrangement (or {\it arrangements})
   into the {\bf arrangements} directory. To see the
  available Cactus arrangements and thorns type
  {\t
    cvs -d :pserver:cvs\_anon@cvs.cactuscode.org:/cactus checkout -s
  }
  To check out an arrangement or thorn go to the arrangements directory,  {\t cd arrangements},
  and  for an arrangement type
{\t
  cvs checkout <arrangement\_name>
  }
        or for just one thorn
{\t
cvs checkout <arrangement\_name/thorn\_name>
}

To simplify this procedure you may use {\t gmake checkout} in the Cactus
home directory which provides menus to pick arrangements and thorns from.


\item[{\bf Update}] To update an existing Cactus checkout (to patch in
  possible changes, etc.), do the following {\em within} the {\tt Cactus} directory.
  {\t
    cvs update
    }
  The update process will operate recursively downwards from your current position
  within the Cactus tree. To update only on certain directories, change
  into these directories and issue the update command.
\item[{\bf CVS status}] To obtain a status report on the ``age'' of your
  Cactus or arrangement routines (from your current directory position
  downward), type
  {\t
    cvs status
    }
\item[{\bf non-anonymous CVS}] If you have an account at the central
  repository ({\tt cvs.cactuscode.org}) and would like to perform
  any of the operation above
  {\em non-anonymously}, replace {\tt cvs\_anon} by your login name
  and provide the appropriate password during the CVS login
  process. Depending on your permissions, you may then make commits to Cactus
  or its arrangements.
\item[{\bf Commits}] You need to perform a personalized login and have
  proper permissions to commit code to the repository.
\end{Lentry}


\chapter{Using TAGS}
\label{sec:usta}
Finding your way around in the Cactus structure can be pretty
difficult to handle. To make life easier there is support for TAGS,
which lets you browse the code easily from within Emacs/XEmacs or vi.
A TAGS database can be generated with by gmake:

\section{TAGS with emacs}
\label{sec:tawiem}

{\tt gmake TAGS} will create a database for a routine reference
table to be used within Emacs. This database can be accessed within
emacs if you add either of the following lines to your {\tt .emacs} file:\\
{\tt (setq tags-file-name "CACTUS\_HOME/TAGS")} XOR \\
{\tt (setq tag-table-alist '(("CACTUS\_HOME" . "CACTUS\_HOME/TAGS")))}\\
where {\tt CACTUS\_HOME } is your Cactus directory.\\

You can now easily navigate your Cactus Flesh and Toolkits by searching for
functions or ``tags'':
\begin{enumerate}
\item \textbf{ Alt.} will find a tag
\item \textbf{ Alt,} will find the next matching tag
\item \textbf{ Alt*} will go back to the last matched tag
\end{enumerate}
If you add the following lines to your {\tt .emacs} file, the
files found with TAGS will opened in {\em read-only} mode:
\begin{verbatim}
(defun find-tag-readonly (&rest a)
  (interactive)
  (call-interactively `find-tag a)
  (if (eq nil buffer-read-only) (setq buffer-read-only t))  )

(defun find-tag-readonly-next (&rest a)
  (interactive)
  (call-interactively `tags-loop-continue a)
  (if (eq nil buffer-read-only) (setq buffer-read-only t))  )

(global-set-key [(control meta \.)] 'find-tag-readonly)
(global-set-key [(control meta \,)] 'find-tag-readonly-next)
\end{verbatim}
The Key strokes to use when you want to browse in read-only mode are:
\begin{enumerate}
\item{CTRL Alt.} will find a tag and open the file in read-only mode
\item{CTRL Alt,} will find the next matching tag in read-only mode
\end{enumerate}

\section{TAGS with vi}
\label{sec:tawivi}

The commands available are highly dependent upon the version of {\bf vi}, but
the following is a selection of commands which may work.


\begin{enumerate}

\item \textbf{vi -t tag}
Start vi and position the cursor at  the  file and line where `tag' is defined.

\item \textbf{Control-]}
Find the tag under the cursor.

\item \textbf{:ta tag}
Find a tag.

\item \textbf{:tnext}
Find the next matching tag.

\item \textbf{:pop}
Return to previous location before jump to tag.

\item \textbf{Control-T}
Return to previous location before jump to tag (not widely implemented).

\end{enumerate}


\chapter{ThornLists}

\label{chap:th}

This section still needs to be written.


%%%%%%%%%%%%%%%%%%%%%%%%%%%%%%%%%%%%%%%%%%%%%%%%%%%%%%%%%%%%%%%%%%%

\end{cactuspart}
